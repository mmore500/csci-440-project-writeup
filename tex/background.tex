
\subsection{The Evolutionary Algorithm}


At the genesis of modern computing, the 1950s, researchers began to apply advancing computational capabilities to investigate and test models of biological evolution. Very quickly they realized the potential of virtual evolution to achieve other ends, setting into motion a line of research that has since blossomed into the field of evolutionary algorithms (EA) design. These algorithms, which use mechanics inspired by biological evolution to evolve novel solutions to a wide array of problems, share a generally consistent basic methodology. The process begins with a population of randomly generated solutions. In a generation-based loop, an elite subset of the population is selected for their fitness (their quality as a solution), subjected to random changes, and recombined with each other to form the next generation. The cycle repeats for as many iterations as desired, and fitness tends to increase with each iteration. Figure \inputandref{working_principle} provides a graphical overview of this process.

When discussing evolutionary algorithms (as well as their biological counterpart) an important distinction is drawn between phenotype and genotype. Phenotype refers to the characteristics of an individual that interact with its environment to determine its fitness. In biology, the physical form of an organism (i.e. its body) is the phenotype. In evolutionary algorithms, the phenotype refers to the characteristics of an individual that are evaluated during selection. Genotype refers to information that is used to determine the phenotype that is passed from generation to generation. In biology, a DNA sequence serves as the genotype. Although many different genotypic encodings are employed in evolutionary algorithms, the genotype ultimately boils down to a collection of digital information.

Researchers and engineers have widely demonstrated the ability of EA to attack labor-intensive optimization problems and to discover novel solutions beyond the reach of human ingenuity \cite{Poli2008AProgramming}. The intervening half century of EA research has seen diversification of the general evolutionary search process described above and diversification of the contents and format of candidate solutions. Today, evolutionary algorithms serve a dual purpose: a tool for biological inquiry and an algorithm for the automatic design of solutions to problems.

\subsection{Plasticity}

Plasticity refers to environmental influence on the phenotype. In biology, environmental and genetic influences, together, determine the phenotype. Environmental influences may alter the trajectory of the developmental process or may otherwise induce phenotype changes in response to environmental stimulus \cite{Fusco2010PhenotypicConcepts}. A conceptual distinction, which will be central to this investigation, can be drawn between direct and indirect plasticity. In the first, environmental influence is exerted directly on developmental or physiological processes. In the second, environmental signals prompt responses that are mediated by physiological or developmental systems; that is, cues from the environment are processed more like informational signals than coercive physical influence \cite{Fusco2010PhenotypicConcepts}. Although the distinction between a signal and coercive influence might appear nebulous at first blush, it has important implications to the design of the proposed experimental regime. At a fundamental level, successful direct plasticity entails \textit{resistance} to environmental influence on the phenotype while successful indirect plasticity entails strategic \textit{amplification} of environmental influence on the phenotype. Figures \inputandref{elephant_developmental_perturbation} and \inputandref{plant_developmental_perturbation} provide a cartoon illustration of this distinction. In Figure \ref{fig:elephant_developmental_perturbation} the cartoon elephant exhibits direct phenotypic plasticity, developing high-fitness phenotypic forms (which, in this example, appear nearly indistinguishable to a casual observer but in general need not be identical) despite variable environmental influence (i.e. diet, temperature, humidity, etc.). In Figure \ref{fig:plant_developmental_perturbation} the cartoon plant exhibits indirect plasticity, developing alternate phenotypic forms in response to variable environmental signals (i.e. light and shadow).

The exact role of phenotypic plasticity in evolution is an issue of active debate in the evolutionary biology community \cite{Pigliucci2008IsEvolvable}. However, several hypotheses describing how phenotypic plasticity might relate to evolution and evolvability have been put forward. Phenotypic plasticity might serve as a kind of local exploration of the phenotypic search space, allowing for the immediate expression of a phenotype with increased fitness and biasing the evolutionary search towards high-fitness regions of the search space \cite{Downing2012HeterochronousBaldwinism}. It is also thought that the homeostatic mechanisms that mediate an organism's interactions with its environment might promote robustness \cite{Moczek2011TheInnovation}. Researchers have suggested that phenotypic modularity might promote plasticity, especially in plants \cite{Schlichting1986ThePlants, DeKroon2005APlants}. Thus, selection for plasticity might promote modularity. In these ways, plasticity might promote useful variability.

Conditional expression of phenotypic traits through plasticity allows for relaxed selection on the genotypic locus determining those traits. Thus, significant genetic variation can accumulate at that locus in a population. In a process known as genetic accommodation, the environmental influence determining when rarely-expressed phenotypic traits are expressed can be diminished or erased through sensitizing mutation; what once was induced via environmental signals can become constitutive. Such processes have been observed experimentally via artificial selection \cite{Moczek2011TheInnovation}.

Finally, plasticity might play a role in concert with indirect genetic encodings, genetic encoding schemes without a direct one-to-one correspondence between genetic information and phenotypic information. Indirect encodings are biased towards phenotypic regularity, \cite{Clune2011OnRegularity} and plasticity might make available otherwise inaccessible phenotypic forms (i.e. providing a mechanism of irregular refinement of highly regular phenotypic structures generated from indirect encodings).

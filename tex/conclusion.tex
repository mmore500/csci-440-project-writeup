The relationship between evolvability and environmental influence on the phenotype was investigated using digital experiments performed on a genetic regulatory model.
The capacity of the model to accommodate the emergence of direct, indirect, and combined plasticity in evolved genetic regulatory networks was confirmed.
The phenotypic response of champion individuals evolved under regimes of direct plasticity and indirect plasticity was assessed.
The model predicts that direct plasticity and indirect plasticity decrease and increase the frequency of silent mutations, respectively.
The model also predicts that combined plasticity induces an increase in the frequency of phenotypically-expressed non-lethal mutation without having a noticeable effect on the observed frequency of silent mutation.

These experimental results confirm the existence of a relationship between phenotypic plasticity and evolvability.
It is hypothesized that this relationship is mediated by internal structural characteristics of the evolved gene regulatory networks.
Specifically, it is postulated that environmental influence on the phenotype induces selection for certain internal characteristics that support direct and/or indirect plasticity which, in turn, affect the outcome of mutation.
The exact nature of internal structural characteristics that support direct and indirect plasticity remains unknown.
Analysis of the outcome of mutation of champion individuals evolved under a combined plasticity regime in comparison to individuals evolved under just a direct plasticity regime and just an indirect plasticity regime suggests that direct and indirect plasticity stem from different aspects of internal structural configuration. 
Further work is called for to pin down a structural characterization of the internal characteristics that support direct and indirect plasticity in order to investigate the hypothesized nature of the relationship between evolvability and phenotypic plasticity.